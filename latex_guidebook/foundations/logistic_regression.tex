\section{Logistic Regression}

\textcolor{blue}{Despite the `regression' bit in the name, logistic regression is a classification model}

\textcolor{blue}{odds ratio\index{odds ratio} (Eq.~\ref{eq:odds_ratio}), where $p$ is representative of the probability of a positive (event we aim to predict) event.}

\begin{equation}
{\frac{p}{1-p}}
\label{eq:odds_ratio}
\end{equation}

\textcolor{blue}{A logit\index{logit} function (Eq.~\ref{eq:logit_def}) is the logarithm of the odds ratio (log-odds)}

\begin{equation}
{logit(p)=\log{\frac{p}{1-p}}}
\label{eq:logit_def}
\end{equation}

\textcolor{blue}{logistic function (sigmoid function) (Eq.~\ref{eq:sigmoid_def}) -- the inverse of a logit function and corresponds to the probability that a certain sample belongs to a particular, positive, class. If the response variable value meats or exceeds the {discrimination threshold}\index{discrimination threshold}, the positive class is predicted.}

\begin{equation}
{S(x)={\frac{1}{1+e^{-x}}}={\frac{e^x}{e^x+1}}}
\label{eq:sigmoid_def}
\end{equation}
