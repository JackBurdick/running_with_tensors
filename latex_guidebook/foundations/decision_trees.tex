%%%%%%%%%%%%%%%%%%%%%%%%%%%%%% Decision Trees
\section{Decision Trees}

\textcolor{blue}{\textcolor{green}{(TODO: revise this para!)} Decision trees make classification decisions based on a series of questions that separate the data into subsets. These questions are chained and result in a tree of questions/decisions where the leaves are considered pure i.e. they contain samples that belong to the same class.}

\textcolor{blue}{Minimum Description Length (MDL) - describes how ``deep'' the tree is allowed to grow.}

\textcolor{blue}{Importance of pruning -- Decision trees can be very deep and can easily lead to overfitting. To help prevent this situation, a limit is set for the maximal depth of a tree. }

\textcolor{blue}{The main advantage to using decision trees is that they are easily interpretable and may often resemble a program developed by hand. Another advantage is that decision trees do not require data to be standardized and they can also tolerate missing feature values.}

\textcolor{blue}{Can be used to determine which features are most useful, support multi-output tasks, and a single decision tree can be used for multi-class classification without having to employ strategies like one versus all.}

\textcolor{blue}{Leaf nodes specify the predicted value of a response variable. When decision trees are used for classification, the leaf nodes represent classes. When used for regression tasks, the leaf nodes are averaged to produce an estimate a response variable.}

% from coursera lec
\textcolor{red}{finding the optimal splitting for a tree is considered an NP-hard problem -- greedy algorithms are used to create splitting schemes that are as close to optimal as possible.}

\subsection{Criterion -- Maximizing Information Gain}

\textcolor{blue}{Term - Information gain -- difference between the impurity of the parent node and the sum of the child node impurities -- the lower the impurity of the child nodes compared to the parent node, the higher the information gain}

\textcolor{blue}{Three commonly used splitting criteria used in binary decision trees: (i) Gini Impurity, (ii) entropy, and (iii) classification error}

\textcolor{blue}{There are no hard rules to determine which criterion to use in a particular instance.}

% see p148[136] of Mastering ML with SKL
\subsubsection{Gini Impurity}

\textcolor{blue}{Measures the proportions of classes in a set}

\textcolor{red}{is greatest when each classes has an equal probability of being selected, like entropy}

\subsubsection{Entropy}

% see p140[128] of Mastering ML with SKL
\textcolor{green}{Entropy is used to quantify the amount of uncertainty}

\paragraph{Information Gain}

% see p144[132] of Mastering ML with SKL
\textcolor{blue}{Information Gain is the measure of reduction in entropy and is the difference between the entropy of the parent node and the weighted average of the children nodes' entropies}

\subsubsection{Classification Error}

\subsubsection{ID3 Algorithm}

\textcolor{blue}{ID3 or ``Iterative Dichotomizer 3'', created by Ross Quinlan.}

\textcolor{blue}{considered greedy --- making locally optimal decisions, not necessarily globally optimal decisions}

\textcolor{blue}{top-down, recursive, depth-first partitioning of the dataset.}

\textcolor{blue}{where each feature is selected because the feature reduces uncertainty in the node more than the other features.}

\textcolor{blue}{Assumes categorical features without any missing values.}

\textcolor{red}{Can be extended to handle continuous descriptive features and continuous target features}

\subsubsection{C4.5 Algorithm}

\textcolor{blue}{Variant of the {ID3 algorithm}\index{ID3 algorithm} that can handle continuous categorical descriptive features and missing features}

\textcolor{blue}{supports pruning}

\textcolor{red}{uses post-pruning}

\textcolor{blue}{{J48}\index{J48} is an open source implementation of the C4.5 algorithm}

\subsubsection{CART Algorithm}

\textcolor{blue}{The CART algorithm is another variant of the ID3 algorithm.}

\textcolor{blue}{Uses the Gini index}

\textcolor{blue}{Can handle continuous target features}

\subsection{Pruning}

\textcolor{blue}{Can help address overfitting. Pruning is an attempt to reduce the size of the tree while still maintaining a similar empirical error.}

\subsubsection{Pre-pruning}

\subsubsection{Post-pruning}

\section{Random Forests}

% see Breiman, 2001
\textcolor{blue}{Ensemble method. Combine various decision trees, where some may be weak learners\index{weak learner} (\textcolor{green}{def}) and some may be strong learners\index{strong learner} (\textcolor{green}{def}). The final classification will be determined by majority vote from the number of trees.}

\textcolor{blue}{Ensemble methods are discussed more in depth in \textcolor{red}{local ref}}



