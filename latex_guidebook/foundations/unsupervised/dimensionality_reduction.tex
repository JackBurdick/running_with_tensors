\section{Dimensionality Reduction}
\label{unsupervised_dimensionality_reduction}

\r{Dimensionality reduction is typically used to reduce the dimensions in a feature representation while retaining as much information as possible.}

\r{reproduce the orginal dataset from teh reduced feature set as well as possible.}

\r{Motivation: i) mitigate issues caused by the curse of dimensionality ii) compress data iii) visualize and explore datasets and improve interpretability -- interpreting data in high dimensions is harder than in lower dimension spaces (particularly three or less). May also be used to compress data before being used by another learning algorithm.}

\r{Example --- projecting 3D data into a 2D space.}

\r{project the raw high-dimensuional input data into  a lower-dimensional space by only \TD{iteratively removing the least explainatory dimensions.}}

\TD{Two major branches of dimensionality --- linear and nonlinear (manifold). The difference is in the projection, one is linear and the other is, you guessed it, nonlinear.}

\r{Unsupervised Transformations --- Dimensionality Reduction --- where the goal is to reduce the dimensionality of the data while retaining as much as the relevant information as possible}

%% topic extraction

\r{A high number of features may be computationally costly. Ability to generalize may be reduced if some of the features capture noise or are irrelevant to the underlying relationship. The goal could be to find the features that account for the greatest changes in the response variable}


\TD{often mention ``saliency'', or the contribution of a given feature to a particular result --- estimated as partial derivative of output with respect to the input, i.e. the feature is considered to have high saliency, if a small change in it (the feature, as an input), greatly alters the output}

\subsection{Principal Component Analysis}

\textcolor{green}{{Principal Component Analysis (PCA)}\index{Principal Component Analysis (PCA)} may also be known as the {Karhunen-Love Transform (KLM)}\index{Karhunen-Love Transform (KLM)} }

\textcolor{red}{eigen-decomposition of the covariance matrix}

%p232[220] of Masstering ML w/SKL
\textcolor{red}{``PCA is most useful whe 'the variance of the dataset is distributed unevenly across the dimensions'' }

% TODO: get var/covar defs (currently in regression tex)
\textcolor{blue}{carvariances between each par of dimensions in a dataset are described in a {covarience matrix}\index{covarience matrix} }

\r{combine highly correlated features -- represent with fewer (linearly uncorrelated) features}

\r{searches for linear combinations in all input variables --- retains as much of the variation (salient information) as possible (some is lost)}

\TD{several variants --- incremental PCA, nonlinear (kernel PCA), sparse (sparse PCA)}

\TD{NOTE: it is important to ensure the features are on the same scale before performing PCA.}

\subsubsection{Linear}

\paragraph{Incremental PCA}

\r{small batches.}

\paragraph{Sparse PCA}

\TD{https://stats.stackexchange.com/questions/305477/criteria-for-choosing-between-pca-and-sparse-pca , https://stats.stackexchange.com/questions/79168/how-exactly-is-sparse-pca-better-than-pca }

\TD{generates PCs slightly differently than normal PCA. Searches for linear combinations in only some of the input variables.}

\subsubsection{Nonlinear}

\paragraph{Kernel PCA}

\TD{more}

\r{similarity funciton (kernel method). Effective when the original feature set is not linear separable}

\TD{type of kernel}

\TD{kernel coefficient (gamma) -- \ALR -- popular radial basis function kernel}


\subsection{Singular value decomposition (SVD)}

\TD{TODO}

\TD{rank matrix}

\subsection{Random Projection}

\TD{TODO}

\textcolor{red}{Johnson-Lindenstrauss lemma}

\r{two versions: standard (gaussian random projection) and sparse (sparse random projection)}

\subsubsection{Gaussian Random Projection}

\r{linear projection}

\subsubsection{Sparse Random Projection}

\TD{TODO}

\section{Nonlinear dimensionality reduction}

%%%% % plus index - unsupervised method
% see p156 of DL for more
\TD{manifold learning (or nonlinear dimensionality reduction)--- {manifold}\index{manifold}, though having a more formal mathematical meaning, will be considered a connected region for our machine learning purposes. --- Nonlinear transormation. PCA and random projection project the data linearly from high to low dimension.}

\subsection{Isomap}

\TD{isomap -- type of manifold learning. estimates the geodesic or curved distance (rather than euclidean) between a point and its neighbors}

\r{relative to neighbors on a manifold \textcolor{red}{rather than a plane?}}

\subsection{Multidimensional Scaling (MDS)}

\TD{TODO}

\subsection{Locally Linear Embedding (LLE)}

\r{segments data into smaller components (neighborhoods), \textcolor{red}{models each component as a linear embedding}}

\r{preserves distance within neighborhoods}

\TD{TODO}

\subsection{t-Distributed Stochastic Neighbor Embedding (t-SNE)}

\TD{TODO}

\TD{t-distributed stochastic neighbor embedding (t-SNE) --- non-linear --- }


\r{two probability distributions, one over pairs of points in a high-dimensional space and another in a low dimensional space. minimizes the \TD{kullback-Leibler divergences} between these two probability distributions.}


\textcolor{red}{nonconvex cost function}

\r{different initializations will generate different results --- no stable solution}

\textcolor{red}{HELLO}

\section{Non-geometric, no distance metric}

\subsection{Dictionary Learning}

\TD{TODO}

\r{learns sparse representation of the original data}

\r{atoms --- vectors in the dictonary. vectors are binary vectors. instances are reconstructed as weighted sum of atoms. easily identify vectors with the most nonzero values}

\r{the dictionary can be undercomplete or overcomplete. undercomplete: atoms $<$ features in the original dataset, or overcomplete: atoms $>$ features in the original dataset.}

\TD{mini-batch version of dictionary learning}


\subsection{Independent Component Analysis (ICA)}

\TD{Separate blended signals into individual components (signal processing)}

\subsection{TODO: others}


% TODO: find a good text on this -- this should be an in depth section
% TODO: dataset for individual voices ina coffeehouse


\TD{Latent Dirichlet allocation}

\r{nonlinear --- multidimensional scaling (MDS), locally linear embedding (LLE), independent componenet analysis (ICA), t-distributed stochastic neighbor embedding (t-SNE), dictionary learning, random trees embedding}

\subsection{Autoencoders}
% TODO: this may not belong here

\TD{discussed more in depth in \ALR}

\r{encoder and decoder --- trained at once}

\r{May be described as a network that learns an approximation of an identity function. reconstructs original features --- hidden layers, reduce parameters (forced to learn salient features), subsequent layers learn increasingly complex relations from the preceeding layers.}

\r{if given too much capacity (``too'' many parameters), the autoencoder may (likely should) simply memorize the observations.}

\paragraph{Undercomplete vs Overcomplete}
\r{undercomplete --- if the constrain encoders output to fewer dimensions than the input}

\r{overcomplete --- when the encoders output dimensions are larger than the input dimensions. typically used in addition to a form of regularization.}

%\paragraph{Sparse vs Dense}



\subsection{Generative Adversarial Networks}
% TODO: this may not belong here

\r{described in more detail in \ref{generative_adversarial_network}}

\subsection{Hidden Markov Model}
% TODO: this may not belong here

\r{simple markov model -- states change stochastically. future states only depend on current state (not prior states)}

\r{Hidden Markov model -- learn propbable next state given what is known about hte sequence of previous states}