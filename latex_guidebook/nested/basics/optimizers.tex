\subsection{Optimizers}

\textcolor{blue}{Estimate the values of the model's parameters that minimize the value of the cost function}

\textcolor{blue}{"turning a loss function into a search strategy"}

\subsubsection{Gradient Descent}

\textcolor{blue}{Gradient Descent --- overview --- optimization algorithm that can be used to estimate the local minimum of a function}

\textcolor{blue}{Iteratively updates the model parameters by calculating the partial derivatives of the cost function at each step during training}

\textcolor{blue}{Gradient descent is only guaranteed to find the local minimum of the cost function.}

\textcolor{blue}{simultaneous update.}


\paragraph{Batch Gradient Descent}

\textcolor{blue}{batch gradient descent --- taking a step (update the weights) opposite (down) the gradient calculated from the entire training set}

\textcolor{blue}{Batch gradient descent is deterministic --- will produce the same paramter values if the same dataset is used multiple times.}


\paragraph{Stochastic Gradient Descent}

\textcolor{blue}{Stochastic Gradient Descent (sometimes called iterative or on-line gradient descent) --- rather than update the weights based on the sum of the accumulated errors, the weights are updated for each training sample}

\textcolor{blue}{Stochastic gradient descent is deterministic --- may produce the different parameter values if the same dataset is used multiple times. May not minimize the cost function as well as gradient descent but the approximation is often ``close enough''.}


\paragraph{Mini-batch Gradient Descent}

\textcolor{blue}{mini-batch gradient descent --- compromise between batch and stochastic gradient descent where the gradient is calculated over a batch of training data}

\textcolor{blue}{Since the gradient is calculated on a single example, the error surface will appear noisier than if it was calculated over a batch or the entire training set.}

\textcolor{blue}{When using stochastic gradient descent, it is important to shuffle the data after each epoch.}


% when looking at specific optimizers, http://ruder.io/optimizing-gradient-descent/ was a useful resource

\subsection{Improved Optimizers}

\textcolor{blue}{Some of the common optimizers are listed below. Additional optimizers are discussed in \textcolor{red}{local ref?}}

\subsubsection{Momentum}

\textcolor{blue}{Momentum~\cite{qian1999momentum}}

\subsubsection{RMSProp}

\subsubsection{Nesterov}

\textcolor{blue}{Nesterov accelerated gradient (NAG)}

\subsubsection{Adam}

\textcolor{blue}{Adaptive Moment Estimation (Adam)~\cite{kingma2014adam}}


%%%%%%%%%%%%%%%%%%%%%%%%%%%%%%%%%%%%%%%%%%%%%%%%%%%%%%%%%%%%%%%%%%%%%%%%%%%%%%%%%%%%%%
%%%%%% TODO: Cut off -- these optimizers will be moved to another location (research?)
%%%%%%%%%%%%%%%%%%%%%%%%%%%%%%%%%%%%%%%%%%%%%%%%%%%%%%%%%%%%%%%%%%%%%%%%%%%%%%%%%%%%%%

\subsubsection{Nadam}

\textcolor{blue}{Nadam (Nesterov-accelerated Adaptive Moment Estimation)~\cite{dozat2016incorporating}}

\subsubsection{Adagrad}

\textcolor{blue}{Adagrad~\cite{duchi2011adaptive}}

\subsubsection{Adadelta}

\textcolor{blue}{Adadelta~\cite{zeiler2012adadelta}}

\subsubsection{AdaMax}




