\subsection{Python}

\textcolor{blue}{This section will not teach how to program in python. Rather, common functionality as well as common XXXXX areas will be introduced.}

\subsubsection{Datatypes}

\paragraph{Tuple}

\textcolor{blue}{fixed-length immutable sequence of Python objects.}

\paragraph{List}

\textcolor{blue}{variable-length mutable sequence of Python objects.}

\paragraph{Dict}

\paragraph{Set}

\subsubsection{Functions}

\paragraph{Built-in Sequence Functions}

\subparagraph{enumerate}

\subparagraph{sorted}

\subparagraph{zip}

\subparagraph{reversed}

\subsubsection{Generators}

\subsubsection{Errors and Exception Handling}

\subsubsection{IO}

\subsubsection{Other}

% {{{00000}}}
\begin{lstlisting}[style=pyInStyle]
for i in ["a","b","c", 'd']:
    print(i)
\end{lstlisting}

\begin{lstlisting}[style=pyOutStyle]
a
b
c
d
\end{lstlisting}
% hello

% sample
\begin{lstlisting}[language=Python]
######################
# Sample code snippet
######################
import numpy as np
 
def incmatrix(genl1,genl2):
    m = len(genl1)
    n = len(genl2)
    M = None #to become the incidence matrix
    VT = np.zeros((n*m,1), int)  #dummy variable
 
    #compute the bitwise xor matrix
    M1 = bitxormatrix(genl1)
    M2 = np.triu(bitxormatrix(genl2),1) 
 
    for i in range(m-1):
        for j in range(i+1, m):
            [r,c] = np.where(M2 == M1[i,j])
            for k in range(len(r)):
                VT[(i)*n + r[k]] = 1;
                VT[(i)*n + c[k]] = 1;
                VT[(j)*n + r[k]] = 1;
                VT[(j)*n + c[k]] = 1;
 
                if M is None:
                    M = np.copy(VT)
                else:
                    M = np.concatenate((M, VT), 1)
 
                VT = np.zeros((n*m,1), int)
 
    return M
\end{lstlisting}

