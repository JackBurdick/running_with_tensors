\section{Numpy}

\textcolor{blue}{Designed to work with homogeneous numerical array data}


\subsubsection{Initialization}

% {{{np_init_01}}}
\begin{lstlisting}[style=pyInStyle]
np_ex = np.array([10,11,12,13,14,15,16,17,18,19,20])
\end{lstlisting}

% {{{np_info_01}}}
\begin{lstlisting}[style=pyInStyle]
print("np_ex = {}".format(np_ex))
print("np_ex.shape = {}".format(np_ex.shape))
print("np_ex.dtype = {}".format(np_ex.dtype))
\end{lstlisting}
\begin{lstlisting}[style=pyOutStyle]
np_ex = [10 11 12 13 14 15 16 17 18 19 20]
np_ex.shape = (11,)
np_ex.dtype = int64
\end{lstlisting}

\subsubsection{Indexing}

\textcolor{blue}{Indexing and slicing numpy arrays is similar to indexing an slicing a python \code{list}.}

% {{{np_index_01}}}
\begin{lstlisting}[style=pyInStyle]
print("np_ex[3] = {}".format(np_ex[3]))
print("np_ex[-1] = {}".format(np_ex[-1]))
\end{lstlisting}
\begin{lstlisting}[style=pyOutStyle]
np_ex[3] = 13
np_ex[-1] = 20
\end{lstlisting}

% {{{np_index_02}}}
\begin{lstlisting}[style=pyInStyle]
print("np_ex[:] = {}".format(np_ex[:]))
print("np_ex[:3] = {}".format(np_ex[:3]))
print("np_ex[3:] = {}".format(np_ex[3:]))
\end{lstlisting}
\begin{lstlisting}[style=pyOutStyle]
np_ex[:] = [10 11 12 13 14 15 16 17 18 19 20]
np_ex[:3] = [10 11 12]
np_ex[3:] = [13 14 15 16 17 18 19 20]
\end{lstlisting}

% {{{np_index_03}}}
\begin{lstlisting}[style=pyInStyle]
print("np_ex[::2] = {}".format(np_ex[::2]))
print("np_ex[::-1] = {}".format(np_ex[::-1]))
print("np_ex[::-3] = {}".format(np_ex[::-3]))
\end{lstlisting}
\begin{lstlisting}[style=pyOutStyle]
np_ex[::2] = [10 12 14 16 18 20]
np_ex[::-1] = [20 19 18 17 16 15 14 13 12 11 10]
np_ex[::-3] = [20 17 14 11]
\end{lstlisting}

\subsubsection{Datatypes}

% {{{np_dtypearg_01}}}
\begin{lstlisting}[style=pyInStyle]
np_ex = np.array([10,11,12,13,14,15,16,17,18,19,20], dtype=np.float64)
\end{lstlisting}
\begin{lstlisting}[style=pyOutStyle]
np_ex.dtype = float64
\end{lstlisting}
\begin{markdown}
Using dtype to specify the type
\end{markdown}


% {{{np_dtype_01}}}
\begin{lstlisting}[style=pyInStyle]
np_ex = np.array([10,11.,12,13,14,15,16,17,18,19,20])
\end{lstlisting}
\begin{lstlisting}[style=pyOutStyle]
np_ex.dtype = float64
\end{lstlisting}
\begin{markdown}
Note what happens if just one is specified as a float
\end{markdown}


\textcolor{blue}{The most common types will be the \code{bool}, \code{int64}, and the \code{float64} }

\begin{tabular}{ r l }
	\code{[int8]} & $(-128, 127)$  \\
	\code{[int16]} & $(-32768, 32767)$  \\
	\code{[int32]} & $(-2147483648, 2147483647)$  \\
	\code{[int64]} & $(-9223372036854775808, 9223372036854775807)$  \\
\end{tabular}

\begin{tabular}{ r l }
	\code{[uint8]} & $(0, 255)$  \\
	\code{[uint16]} & $(0, 65535)$  \\
	\code{[uint32]} & $(0, 4294967295)$  \\
	\code{[uint64]} & $(0, 18446744073709551615)$  \\
\end{tabular}

\begin{tabular}{ r l }
	\code{[float16]} & half precision  \\
	\code{[float32]} & single precision  \\
	\code{[float64]} & double precision  \\
\end{tabular}

\subsubsection{Special Initialization}

\paragraph{zeros}

% {{{np_zeros_01}}}
\begin{lstlisting}[style=pyInStyle]
np_ex = np.zeros(shape=(3,6))
\end{lstlisting}
\begin{lstlisting}[style=pyOutStyle]
np_ex = 
[[0. 0. 0. 0. 0. 0.]
 [0. 0. 0. 0. 0. 0.]
 [0. 0. 0. 0. 0. 0.]]
np_ex.shape = (3, 6)
np_ex.dtype = float64
\end{lstlisting}

\paragraph{ones}

% {{{np_ones_01}}}
\begin{lstlisting}[style=pyInStyle]
np_ex = np.ones((3,6))
\end{lstlisting}
\begin{lstlisting}[style=pyOutStyle]
np_ex = 
[[1. 1. 1. 1. 1. 1.]
 [1. 1. 1. 1. 1. 1.]
 [1. 1. 1. 1. 1. 1.]]
np_ex.shape = (3, 6)
np_ex.dtype = float64
\end{lstlisting}

\paragraph{empty}

% {{{np_empty_01}}}
\begin{lstlisting}[style=pyInStyle]
np_ex = np.empty((2,2))
\end{lstlisting}
\begin{lstlisting}[style=pyOutStyle]
np_ex = 
[[2.72876563e+114 5.58294180e-322]
 [4.66005481e-310 4.66005485e-310]]
np_ex.shape = (2, 2)
np_ex.dtype = float64
\end{lstlisting}
\begin{markdown}
The output will vary
\end{markdown}

\paragraph{eye}

% {{{np_eye_01}}}
\begin{lstlisting}[style=pyInStyle]
np_ex = np.eye(3)
\end{lstlisting}
\begin{lstlisting}[style=pyOutStyle]
np_ex = 
[[1. 0. 0.]
 [0. 1. 0.]
 [0. 0. 1.]]
np_ex.shape = (3, 3)
np_ex.dtype = float64
\end{lstlisting}
\begin{markdown}
Identity matrix
\end{markdown}


\paragraph{full}

% {{{np_full_01}}}
\begin{lstlisting}[style=pyInStyle]
np_ex = np.full((3,3), 9)
\end{lstlisting}
\begin{lstlisting}[style=pyOutStyle]
np_ex = 
[[9 9 9]
 [9 9 9]
 [9 9 9]]
np_ex.shape = (3, 3)
np_ex.dtype = int64
\end{lstlisting}
\begin{markdown}
Constant
\end{markdown}

\paragraph{asarray}

\textcolor{blue}{\code{asarray} will convert a python \code{list} to an numpy array}

% {{{np_asarray_01}}}
\begin{lstlisting}[style=pyInStyle]
some_list = [1,2,3,4,5,6,7]
list_as_arr = np.asarray(some_list)
\end{lstlisting}
\begin{lstlisting}[style=pyOutStyle]
some_list: [1, 2, 3, 4, 5, 6, 7]
list_as_arr: [1 2 3 4 5 6 7]
list_as_arr.shape: (7,)
\end{lstlisting}

\paragraph{arange}

% {{{np_arange_01}}}
\begin{lstlisting}[style=pyInStyle]
vals_1D = np.arange(27)
\end{lstlisting}
\begin{lstlisting}[style=pyOutStyle]
vals_1D = 
[ 0  1  2  3  4  5  6  7  8  9 10 11 12 13 14 15 16 17 18 19 20 21 22 23
 24 25 26]
vals_1D.shape = (27,)
\end{lstlisting}
\begin{markdown}
.arange will create a 1D array in the range [0,value)
\end{markdown}



\subsubsection{Reshape}

\textcolor{blue}{\code{reshape} can be used to reshape a matrix.}

% {{{np_reshape_01}}}
\begin{lstlisting}[style=pyInStyle]
vals_2D = vals_1D.reshape(9,3)
\end{lstlisting}
\begin{lstlisting}[style=pyOutStyle]
vals_2D = 
[[ 0  1  2]
 [ 3  4  5]
 [ 6  7  8]
 [ 9 10 11]
 [12 13 14]
 [15 16 17]
 [18 19 20]
 [21 22 23]
 [24 25 26]]
vals_2D.shape = (9, 3)
\end{lstlisting}


% {{{np_reshape_02}}}
\begin{lstlisting}[style=pyInStyle]
vals_2D = vals_1D.reshape(3,9)
\end{lstlisting}
\begin{lstlisting}[style=pyOutStyle]
vals_2D = 
[[ 0  1  2  3  4  5  6  7  8]
 [ 9 10 11 12 13 14 15 16 17]
 [18 19 20 21 22 23 24 25 26]]
vals_2D.shape = (3, 9)
\end{lstlisting}


% {{{np_reshape_03}}}
\begin{lstlisting}[style=pyInStyle]
vals_3D = vals_1D.reshape(3,3,3)
\end{lstlisting}
\begin{lstlisting}[style=pyOutStyle]
vals_3D = 
[[[ 0  1  2]
  [ 3  4  5]
  [ 6  7  8]]

 [[ 9 10 11]
  [12 13 14]
  [15 16 17]]

 [[18 19 20]
  [21 22 23]
  [24 25 26]]]
vals_3D.shape = (3, 3, 3)
\end{lstlisting}


\subsection{Arithmetic}

\subsubsection{Basic}

\subsubsection{Statistical Methods}

\subsection{IO}

% save

% savez

\subsection{Other}

\subsubsection{transpose}

\subsubsection{Set Logic}

% unique


