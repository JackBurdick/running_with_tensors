\chapter{Basics}

\section{Overview}

%%%%%%%%%%%%%%%%%%%%%%%% obligatory "No Free Lunch"
% TODO: there is more to the no free lunch than simply this.
\r{I'm not sure it's possible to discuss machine learning without at least mentioning the ``No Free Lunch'' theorem, which states ``No single classifier works best across all possible scenarios''~\cite{wolpert1997no}}

% “You can’t learn from data without making assumptions”
\TD{A Lot to unpack here.}
\TD{https://peekaboo-vision.blogspot.com/2019/07/dont-cite-no-free-lunch-theorem.html}

%%%%%%%%%%%%%%%%%%%%%%%% Types of data
\r{categorical or numerical. Numerical can be discrete or continuous}

%%%%%%%%%%%%%%%%%%%%%%%% Measurement Levels
\r{qualitative or quantitative.} 

\r{Qualitative can be nominal (aren't numbers and can't be put in any order -- e.g. the seasons: spring, summer, fall, winter) or ordinal (groups and categories that follow a strict order -- e.g. difficult levels: hard, medium, or easy)}

\r{Quantitative are represented by numbers but can be interval (0 is meaningless -- e.g. temperature in C or F, where true zero is not 0) or ratio (has a true 0 -- e.g. temperature in K, weight or length)}

\section{Workflow Overview / Blue Print}

% TODO: does not belong here
\r{feature extraction -- (manual or learned) == ``classification''}

% TODO: this will need to be checked+rechecked+redone
\r{1. explore, 2. create datasets, 3. benchmark}
\begin{enumerate}[noitemsep,topsep=0pt]
	\item Problem definition
	\item Hypothesis?
	\item explore data
	\item (shuffle? representative)
	\item remove split for testing
	\item prepare data
	\item split into train/val
	\item Choose measure of success
	\item Perform baseline -- what performance is expected/realistic goal expectations/ how does a simple, well tested, classifier work?
	\item Develop model
	\begin{enumerate}[noitemsep,topsep=0pt]
		\item Can you (over)fit the training data? - more data?
		\item Fit the validation as best as possible (regularization, augmentation) (WARN: \textcolor{red}{local ref to information leak})
	\end{enumerate}
	\item OTHERS...
	\item Evaluate
\end{enumerate}


\section{Some Terms}

\emph{input variable(s)} -- predictors, independent variables, features, regressors, controlled variables, exposure variables or simply variables.
 
\emph{output variable(s)} -- response or dependent variable. May also be known as regressands, criterion variables, measured variables, responding variables, explained variables, outcome variables, experimental variables, labels.

\textcolor{blue}{Both input and output variables may take on continuous or discrete values.}

\emph{relationship} $Y = f(x) + \epsilon$ \textcolor{blue}{estimate $f$. prediction and inference}.

\textcolor{blue}{\emph{{reducible error}\index{reducible error}} -- the estimated function $\hat{f}$ will likely not be perfect, and the reducible error is the error that could potentially be corrected (perhaps by using a more appropriate learning technique to estimate $\hat{f}$).  The \emph{irreducible error} is an error that can not be corrected. The irreducible error may be larger than zero due to either \emph{unmeasured variables} \emph{e.g.} variables that were not measured or \emph{unmeasurable variation} \emph{e.g.} an individual's feelings/emotions or variation in the production of a product, or both. The irreducible error provides an upper bound on the performance of the predicted $\hat{f}$}

\textcolor{red}{inference: relationship between predictors and response}

% TODO: not sure ~exactly where this fits yet..
\textcolor{blue}{parameters -- model variables that change during training. Hyper-parameters are set before training.}

\section{Type of Learning}

\r{Three main types of machine learning: supervised, unsupervised, and reinforcement. Semi-supervised learning is yet another area of machine learning, which lies somewhere between supervised and unsupervised learning.}

% From ML for Predictive Data Analytics
\textcolor{blue}{Another way to group types of learning -- Information-based, similarity-based, probability-based, and error-based}



\begin{itemize}[noitemsep,topsep=0pt]
	\item Supervised Learning (\ALR)
	\begin{itemize}[noitemsep,topsep=0pt]
		\item \r{observe input variables with corresponding output values. A program that predicts an output for an input by learning from pairs of labeled inputs and outputs. Classification \ALR and regression \ALR are subcategories of supervised learning}
	\end{itemize}
	\item Unsupervised (\ALR)
	\begin{itemize}[noitemsep,topsep=0pt]
		\item \r{observe input variables without corresponding output values and attempts to discover patterns in the data. There is no error signal to measure, rather, performance metrics report some attribute of structure discovered in the data, such as the distances within and between clusters. Determining whether the method learned ``something'' useful is inherently difficult --- since, by definition, there are no labels.}
	\end{itemize}
	\item{Reinforcement (\ALR)}
	\begin{itemize}[noitemsep,topsep=0pt]
		\item \r{Reinforcement learning does not learn from labeled pairs of inputs and outputs, rather it learns from `feedback' from decisions that are not explicitly corrected. Information is still supplied to the system as to whether the networks outputs are good or bad, but no actual desired values are given. Goal -- develop an \emph{agent} that improves it's performance based on interactions with an \emph{environment} based on a \emph{reward}}
	\end{itemize}
	\item Semisupervised Learning (\ALR)
	\begin{itemize}[noitemsep,topsep=0pt]
		\item \r{makes use of both supervised and unsupervised data.}
	\end{itemize}
\end{itemize}


\subsection{Supervised vs Unsupervised}

\subsection{Classification vs Regression}

\subsubsection{Regression} 

Regression, also called regression analysis \textcolor{red}{local ref?} involves predicting a continuous or quantitative output value. For example attempting to find a relationship between a given predictor/explanatory variables (age, job title, zip code) and a continuous response (an individuals outcome).

\subsubsection{Classification} 

Classification involves predicting categorical (discrete) or qualitative output value (such as a non-numerical value). 

\textcolor{blue}{Binary classification (benign vs malignant) and multi-class classification (identifying many different skin diseases).}

\subsection{Multi-class}

% TODO:
\begin{itemize}[noitemsep,topsep=0pt]
	\item \TD{macro averaged: all classes independently, averaged}
	\item \TD{micro averaged:}
	\item \TD{weighted: macro, weighted by sample frequency}
\end{itemize}


\subsection{Multi-label}
\TD{TODO: {multi-label classification}\index{multi-label classification} --- where a classifier assigns multiple labels to each instance}


% TODO: placement
\textcolor{blue}{Evaluating every output node for every example can quickly become computationally expensive. approximate version of softmax. i) candidate sampling (tf.nn.sampled\_softmax\_loss()) \textcolor{red}{calculates the output for all of the positive labels, but will only calculate the label for a random sample of negatives. Where the number of negatives sampled is a a hyperparameter} ii) noise-contrastive estimation (tf.nn.nce\_loss()) \textcolor{red}{approximates the denominator of softmax by modeling the distribution of outputs}. Typically, these two methods may be used during training, but the full softmax function will be used during inference.}


\subsubsection{Approaches: Problem transformation}

\textcolor{blue}{There are two main approaches to multi-label classification}

\textcolor{blue}{{Problem transformation}\index{Problem transformation} modify the original multi-label problem to a set of single-label classification problems.}

\paragraph{Unique set/combination of labels}

\textcolor{green}{TODO: table and example.}

\textcolor{blue}{Two main concerns with this methodology: i) increasing the number of classes is impractical and will often have very few instances and ii) the classifier can only predict combinations that were seen in the training data.}

\paragraph{Many Binary Classifiers}

% p124[112] of Mastering ML with SKL
\textcolor{blue}{Train a classifier for each label in the training set. The final prediction is the combination of all the predictions from the binary classifiers.}

\textcolor{blue}{The main concern with this approach is that the relationships between labels is ignored.}

\subsubsection{Evaluating Multi-label Classification}

\textcolor{blue}{see \textcolor{red}{local ref}.}

% page 94 of AGtext
One-versus-all \emph{OvA} (also \emph{one-versus-rest}) -- $n$, separate binary classification problems, where the $n$ is the number of classes. The target class may be assigned as the positive class and `all' the `rest' of the classes may be assigned as the negative class.

One-versus-one (OvO) -- train a binary classifier for every pair


\textcolor{blue}{Binary classification can be extended to multi-class classification via the OvR method.}

%\subsubsection{Bayes Classifier}

\section{Training}

\subsection{Cost, Loss Function}

\textcolor{red}{Cost and Loss functions -- I'm not sure why I didn't have this yet?}

\textcolor{blue}{A loss function is responsible for providing a measure of performance for the training data -- thus the loss function is responsible for guiding the training process.}

\textcolor{blue}{\textit{objective function}}

\textcolor{red}{example of plots - step by step}

\textcolor{red}{contour maps}

\textcolor{blue}{it is possible that a DNN graph may have multiple loss functions (where each one is responsible for a specific output class). Since the gradient descent process relies on a single scalar loss value, graphs with multiple losses will combine these losses (\textcolor{red}{via averaging})}

\textcolor{blue}{TODO: general loss plot example/figure}

\textcolor{blue}{TODO: section (maybe in another location) that shows how to "troubleshoot" a loss plot. So examples of what a noisy loss plot vs a flat loss plot might indicate. Would also be nice to expand to include examples of the loss plot of the training and validation (likely another section, but with a reference in this section pointing to it.)}

\textcolor{blue}{TODO: talk about "convexity" -- show 3D loss plot (2D for params, 3rd for loss), and explain how running/re-running a model may result in slightly different values.}

\textcolor{blue}{Loss calculation. TODO: example for linear linear regression and why sum of loss will cancel out terms. \textcolor{red}{point to local ref.}}

\r{``surrogate loss'' --- takes a classification problem and turns it to a continuous/smooth surface.}

\subsection{Error Functions}

\TD{discussion of techniques used for training networks}

\r{the training of a network consists of a suitable error function to be minimized with respect to the parameters (weights and biases) of a network.}



\subsubsection{forward pass}

\subsubsection{backward pass}

\subsubsection{Least-squares techniques}

\r{simple error function, typically most suitable for regression problems, but could also be used for classificaiton problems, though other error functions typcially perform better for classification (\textcolor{red}{local ref})}

\paragraph{Sum-of-squares error function}

\r{potential negative --- the largest contributions are from points with the largest error. If the dataset contains many outliers (or if the distribution contains long tails), the the adjustments can be dominated by these outliers. NOTE: in this case especially, it is essential to ensure there are no mislabeled data, as these can have dramatic effects. Techniques used to address this problem are known as robust statistics \TD{local ref -- create subsec for this == \textcolor{red}{see (Huber 1981)}}}

% see p.211 of NN by bishop (p.226 on tablet)
\textcolor{red}{input-dependent variance}

\begin{figure}[htp]
	\centering
	\includegraphics[width=0.4\textwidth]{example-image-a}\hfil
	\includegraphics[width=0.4\textwidth]{example-image-b}\hfil
	\caption{ \TD{figure of how a model is affected by outliers. left: the data does not have any outliers and a linear line fits the data fairly well. right: a single outlier dramatically alters the linear line}}
	\label{fig:basics_error_fn_sumofsquares_outlier}
\end{figure}

\r{can be used for regression or classification (though typically for regression)}

 
\paragraph{Normal Equations}

\r{find an explicity solution to the function}

\paragraph{Singular Value Decomposition (SVD)}
% TODO: read: https://twitter.com/WomenInStat/status/1285610321747611653

\r{Singular Value Decomposition (SVD) --- technique used to help solve problems like ``near degeneracies'' (\textcolor{green}{TODO: unsure of this --- also see Press et al. 1992 for an introduction})}


\paragraph{Gradient Descent}

\r{repeatedly choosing and moving toward a descent direction until convergence} \TD{local descent may be another name}. \TD{there are many schemes for choosing the size of the step, discussed further in \ALR.}

%\r{\textcolor{green}{TODO: is this the first time I've talked about this?}. Finding the weight values of a sum-of-squares error function can be found explicitly in terms of the \TD{pseudo inverse of a matrix} (if a linear network).}
% However, if a non-linear activation fuction is used, then the closed form of a solution is no longer possible ? % what is the source for this?

\r{If a derivative of the activation function is differentiable, the derivatives of the error function with respect to the weight parameters can then be evaluated. The derivatives can then be used by gradient-based optimization algorithms to find the minimum of the error function \TD{local ref - sec on optimizers}}

\r{update the parameters one iteration at a time --- sequential, pattern-based, update}


\subsubsection{Global vs Local Minima}

\ALR to calculus section

% TODO: this needs to be placed elsewhere
\textcolor{blue}{show how a simple problem may get "stuck" in a local mimima (plot decision boundary for 2D feature space) and show how it changes with different initialization (use different random seeds). }

\textcolor{blue}{{inappropriate minima}\index{inappropriate minima} -- don't reflect the relationship between features and output and/or don't generalize well.}

\r{a local minimum will have a zero derivative, as it is not possible for a nonzero derivative to be a mimima.  However, just because a point has a nonzero derivative, it does not mean that it is a minima, it could also be local maxima or an inflection point. A positive second derivative would indicate that the point is a local minima.}

\section{Quality of Fit}

%% regression example

\subsection{Regression Example}

% TODO: need to think about placement of these and how to organize the general case.

\textcolor{blue}{squared will handle the issue with positive and negative error terms canceling each other out as well as penalizing terms more when they are larger}

\textcolor{blue}{Mean Squared Error.$\hat{f}(x_i)$ is the prediction that $\hat{f}$ produces for the $i$th sample (\ref{eq:MSE_def}). The output will be small for predicted values that are similar to the ground truth}

\begin{equation}
{MSE = \frac{1}{n}\sum_{i=1}^{n}(y_i - \hat{f}(x_i))^2}
\label{eq:MSE_def}
\end{equation}

\r{The MSE may be hard to interpret since the error is squared. \textcolor{red}{example}}

\r{Root mean squared error, which is simply the root of the MSE. (\ref{eq:RMSE_def})}

\begin{equation}
{RMSE = \sqrt{\frac{1}{n}\sum_{i=1}^{n}(y_i - \hat{f}(x_i))^2}}
\label{eq:RMSE_def}
\end{equation}

\textcolor{blue}{"problem" with this loss function is that it does not follow the intuition that really bad predictions should be penalized more harshly than predictions that are just "a little bad"}

% TODO: again, need to devise a better way to organize this section (loss functions)
\textcolor{red}{Cross entropy or log loss. \textcolor{red}{CITE} (related to Shannon's information theory \textcolor{red}{CITE})}

% TODO, include in losses?: penealize highly confident, highly incorrect
\TD{Log loss: def + interpretation}


%% classification example

\subsection{Classification Example}

\textcolor{blue}{The proportion of mistakes that are made.}

\begin{equation}
{error\_rate = \frac{1}{n}\sum_{i=1}^{n}(y_i \ne \hat{y_i})}
\label{eq:class_error_rate_def}
\end{equation}

\textcolor{blue}{$\hat{y_i}$ is the predicted classification label for the $i$th observation using our predictor/model $\hat{f}$ and $y_i$ is the ground truth label}

\section{Describing Learners}

\subsection{Parametric and non-parametric}

\subsubsection{parametric}

\rr{parametric models are models that learn a fixed number of parameters that are able to classify new data points without requiring the original dataset anymore. First, a function form is selected (linear, polynomial, etc.), then the coefficients for the function are learned from the training data.}

\textcolor{green}{TODO: example \r{predicting the income of an individual $income \approx \beta_0 + \beta_1 \times education_{yrs} + \beta_2 \times experience_{yrs}$ --- assuming a linear relationship between response and two predictors}}

\textcolor{green}{TODO: plot of example -- }

\begin{figure}[htp]
	\centering
	\includegraphics[width=0.5\textwidth]{example-image-a}\hfil
	\caption{Figure example of assumed linear model and datapoints \textcolor{green}{TODO}}
	\label{fig:basics_para_assume_linear}
\end{figure}



\textcolor{blue}{Examples of parametric models may be simple artificial neural networks, naive bayes, logistic regression, etc.}

\subsubsection{nonparametric}

%% unsure about this! 
\textcolor{red}{Nonparametric models are not models without parameters, rather they are models were the number of parameters are not fixed, they may grow with the number of training instances}

\textcolor{blue}{May be useful when little is known about the underlying relationship in the data and there is an abundance of data.}

\textcolor{blue}{An Example of a nonparametric model may be k-Nearest neighbors -- where the model does not assume anything about the form of the mapping function and makes predictions based on the k most similar training instances.}

\subsubsection{parametric vs nonparametric}

\r{An advantage of a nonparametric approach may be that the model does not make any explicit assumptions about the best fitting model thus avoiding limiting the model to a functional form $f$ that may not be similar to the true $f$ --- for example, using a linear model for a model that is cleary \textcolor{red}{parametric} in form. Typically, since a nonparametric approach is not limited to an explicit number of parameters, a larger amount of data is required to obtain an accurate estimate of $f$.}

\textcolor{blue}{A disadvantage to this type of approach is that the computational complexity for classifying new samples grow linearly with the number of samples in the training set.}



\subsection{Eager vs Lazy Learners}

\textcolor{green}{TODO: Eager vs Lazy overview}
\textcolor{blue}{Training an eager learner is often more computationally expensive, but typically prediction with the resulting model is inexpensive.}

\subsubsection{Eager Learners}

\textcolor{blue}{Eager learners estimate the parameters of a model that generalize to a training set --- build an input-independent model}

\subsubsection{Lazy Learners}

\r{Also known as Instance-based Learners}

\r{do not spend time training, but may predict responses slowly (relatively) compared to eager learners}

\r{Lazy learners store the training dataset with little to no processing.}


\subsection{Generative vs Discriminative Models}

\TD{TODO: Generative vs Discriminative models overview}


\subsubsection{Discriminative Models}

\r{learn a decision boundary that is used to \textit{discriminate} between classes. There exist both probabilistic and non-probabilistic discriminative models.}

\paragraph{Probabilistic Discriminative}

\r{Probabilistic discriminative models learn to estimate the conditional probability i.e. which class is most probable given the input features.}

\paragraph{Non-probabilistic Discriminative}

\r{Non-probabilistic discriminative models directly map features to classes.}

\subsubsection{Generative Models}

% see p129[117] of Mastering ML w/SKL
\TD{TODO: Generative Models --- ``do not learn a decision boundary, rather, they model the joint probability distribution of the features and classes i.e. they model how the classes generate features. Then, using Bayes' theorem, they are able to estimate the conditional probability of a class given the features.''}

\r{must be probabilistic, not deterministic and also some degree of randomness (otherwise the same output would be generated each time).}

\TD{TODO: will need to expand into a much larger section and talk about some types of generative models}

\TD{TODO: ``important'' examples --- GPT2, StyleGAN}

\TD{TODO: use cases -- direct and indirect. music, art, game design, simulations for RL (paper\cite{ha2018world})}

\r{\TD{Generative Deep Learning} makes a point that (paraphrasing) ``categorizing data is not enough, we should try to understand how and why the data came to existence in the first place.''}


% see p130[118] of Mastering ML w/SKL
\r{One advantage of generative models is that they can be used to generate new examples of data}

\subsection{Strong vs Weak Learners}

\TD{TODO: Strong vs Weak learners (classifier, predictor, etc.) overview}
%\textcolor{blue}{}

\begin{itemize}[noitemsep,topsep=0pt]
	\item Strong Learners
	\begin{itemize}[noitemsep,topsep=0pt]
		\item \r{Strong Learners are models that are arbitrarily better than weak learners.}
	\end{itemize}
	\item Weak Learners
	\begin{itemize}[noitemsep,topsep=0pt]
		\item \r{Models (typically simple models) that perform only slightly better than random chance. Typically used in ensemble methods (discussed in more detail in \ALR)}
	\end{itemize}
\end{itemize}

\section{Online Learning}

% See p.246 of Understanding Machine learning
\textcolor{blue}{difference to \textcolor{red}{PAC learning?}}


\section{Kernel Methods}
\label{sec:kernel_trick}

\r{Adding non-linear features to data in attempt linearly separate the data.}

 %Adding non-linear features is a powerful method for allowing linear methods to separate non-linear data. However, which features, combinations of features, and types of features is often not easily known. And adding may of these features may become computationally limiting

\r{Transform the training data onto a higher dimensional feature space}

% see p177[165] of mastering ML with SKL

\r{Choosing an appropriate kernel can be challenging}

% computing the distance (scalar products) of data points for the expanded feature representation --- but doesn't compute the expansion

% see p180[168] of mastering ML w/SKL for more on kernels
\r{Some commonly used kernels include polynomial, RBF, sigmoid, Guassian, and linear kernels}

\r{commonly used in SVMs (see \textcolor{red}{local ref}), the kernel trick can be used with any model that can be expressed in terms of the dot product of two feature vectors.}

\r{the mapping function is not fully computed due to the kernel trick}

\subsection{Kernel Trick}

\subsection{Kernels}

\r{There are many kernel functions. Choosing the ``best'' kernel will depend on the current problem.}

\textcolor{red}{kernel functions are continuous and symmetric}

\textcolor{red}{TODO: put these into context and/or give an example}

\textcolor{blue}{the word kernels is representative of a weighting function \textcolor{red}{or weighted sum or integral}}


% https://data-flair.training/blogs/svm-kernel-functions/
% http://crsouza.com/2010/03/17/kernel-functions-for-machine-learning-applications/

\textcolor{red}{TODO: automatic kernel selection}

\subsubsection{Polynomial}

\r{polynomial computes all possible polynomials of the original features up to a certain degree}

\textcolor{blue}{parameters: slope ($\alpha$), constant $c$, polynomial degree $d$}
\begin{equation}
{k(x, y) = (\alpha x^T y + c)^d}
\label{eq:kernel_polynomial_eq}
\end{equation}
% TODO: double check

\subsubsection{Gaussian / RBF (Radial Basis Function)}

\textcolor{blue}{circles/hypersphere}
\textcolor{red}{infinite-dimensional feature space}

\begin{equation}
{k(x, y) = exp(- \gamma || x_1 - x_2 || ^2 ) }
\label{eq:kernel_guassian_rbf_eq}
\end{equation}

\textcolor{blue}{$|| x_1 - x_2 ||$ represents the euclidean distance and $\gamma$ represents the parameter that controls the width of the Gaussian kernel (the inverse width of the Gaussian kernel). $\gamma$ controls how far the influence of a single training instance reaches --- high values correspond to a limited reach (typically result in lower complexity) and low values correspond to a far reach (typically result in higher complexity).}

\textcolor{green}{Would be nice to have a figure with low - medium - high values for the hyperparameters and the outcome}


\subsection{Less Common Kernels}



%\subsubsection{Laplace RBF}

%\subsubsection{Hyperbolic Tangent}

%\subsubsection{ANOVA Radial Basis Kernel}

%\subsubsection{Sigmoid}

%%%%%%%%%%%%%%%%%%% Hyper-parameters
\section{Hyper-Parameters}

\subsection{Parameters: "tuning knobs"}

\subsubsection{Learning Rate}

\textcolor{green}{TODO: Learning rate overview}

% TODO: Learning rate practical advice

% TODO: figure showing cost vs iteration for a LR that is too small, just right, and too large

% TODO: Learning rate figure showing how if the learning rate is too high, you'll likely see the cost diverage when plotted vs iterations

\r{In general, if the LR is too small, convergence (with something like gradient descent) may be slow.  If LR is too large, then convergence may not occur and the reduction in error may oscillate wildly or may even diverge.}

\paragraph{research}

% TODO: this section may not belong here - may belong in an "advanced section"

%%%% learning rates
\textcolor{blue}{cyclic learning rate~\cite{smith2017cyclical}}

\textcolor{blue}{sgdr: stochastic gradient descent with restarts~\cite{loshchilov2016sgdr}. The learning rate is decreased from the max value along a curve (cosine, shown in Eq.\ref{eq:sgdr_def}, where $n_{max}^i$ and $n_{min}^i$ are ranges for the learning rate, $T_i$ represents epochs, $T_{cur}$ is how many epochs have been performed since the last restart). The authors also suggest making each next cycle longer than the previous cycle by a constant $T_mul$ may be beneficial.}

\begin{equation}
{n_t = n_{min}^i + 1/2(n_{max}^i - n_{min}^i)(1 + cos(\frac{T_{cur}}{T_i}\pi))}
\label{eq:sgdr_def}
\end{equation}

\subsubsection{Batch size}

\textcolor{green}{TODO: batch size overview}

\textcolor{blue}{optimal batch size is problem dependent}

\textcolor{blue}{TODO: notebook and plots showing how the smoothness is affected when comparing batch sizes of 1 vs 10 vs 20 etc.}

% related to shuffling - the gradients are computed on a batch and so a batch should be representative of the data

%%%%% small batch size
\textcolor{blue}{small minibatch sizes (between 2 and 32) may be better than large batch sizes~\cite{masters2018revisiting}.}

\textcolor{blue}{``generalization gap'' may not be due to large mini-batches, rather, due to the number of updates made to the system~\cite{hoffer2017train}}

%%%% minibatch
\textcolor{blue}{Batch training is almost always slower to converge than on-line/mini-batch training, which is likely due to the fact that on-line/mini-batches learning will follow the error surface, allowing for larger learning rates, and thus faster convergence~\cite{wilson2003general}.}

% incrementing batchsize over time
\textcolor{blue}{Increasing the batch size may achieve similar benefits to decaying the learning rate ~\cite{smith2017don} -- which could lead to use of larger batch sizes, reducing the number of parameter updates and therefore reducing training time.}

\subsection{Hyper-Parameter Optimization}

\subsubsection{Grid Search}

\textcolor{blue}{{Grid search}\index{Grid search} Exhaustive search that trains+evaluates a model for each combination of specified hyperparameter configurations and combinations defined by a Cartesian product of the sets of possible values for each hyperparameter.}

\subsubsection{Randomized Search}

\textcolor{blue}{{Randomized search}\index{Randomized search} }

\textcolor{green}{TODO: figure demonstrating difference between grid and randomized search}

\subsubsection{Other Methods}

\textcolor{blue}{See \textcolor{red}{local ref? --- advanced methods and research}}

%%%%%%%%%%%%%%%%%%%%%%%% Optimizers

\chapter{Estimating Model Parameters}

\textcolor{green}{Iterative Estimation vs Calculation}

% http://mathworld.wolfram.com/NormalEquation.html
% https://eli.thegreenplace.net/2014/derivation-of-the-normal-equation-for-linear-regression
\textcolor{green}{TODO: Normal Equation}

% TODO: non-invertable (singular or degenerate) matrix

% common causes (not verified) - 1) redundant features 2) too many features - more features than samples. solutions may be to delete features

\subsection{Initialization}

\TD{TODO: initialization methods and importance}

\TD{figure showing the importance of initialization strategies for different architectures after \textit{n} layers}

\subsubsection{Strategies}

\TD{TODO: strategies overview}

\paragraph{glotrot}

\paragraph{he}

\subsubsection{Implementation}

\TD{TODO: different types of parameters may benefit from different strategies}

\paragraph{Weights}

\TD{TODO: fully connected, convolution}

\paragraph{Biases}

\chapter{Optimization}

\subsection{Optimizers}

\textcolor{blue}{Estimate the values of the model's parameters that minimize the value of the cost function}

\textcolor{blue}{"turning a loss function into a search strategy"}

\subsubsection{Gradient Descent}

\textcolor{blue}{Gradient Descent --- overview --- optimization algorithm that can be used to estimate the local minimum of a function}

\textcolor{blue}{Iteratively updates the model parameters by calculating the partial derivatives of the cost function at each step during training}

\textcolor{blue}{Gradient descent is only guaranteed to find the local minimum of the cost function.}

\textcolor{blue}{simultaneous update.}


\paragraph{Batch Gradient Descent}

\textcolor{blue}{batch gradient descent --- taking a step (update the weights) opposite (down) the gradient calculated from the entire training set}

\textcolor{blue}{Batch gradient descent is deterministic --- will produce the same paramter values if the same dataset is used multiple times.}


\paragraph{Stochastic Gradient Descent}

\textcolor{blue}{Stochastic Gradient Descent (sometimes called iterative or on-line gradient descent) --- rather than update the weights based on the sum of the accumulated errors, the weights are updated for each training sample}

\textcolor{blue}{Stochastic gradient descent is deterministic --- may produce the different parameter values if the same dataset is used multiple times. May not minimize the cost function as well as gradient descent but the approximation is often ``close enough''.}


\paragraph{Mini-batch Gradient Descent}

\textcolor{blue}{mini-batch gradient descent --- compromise between batch and stochastic gradient descent where the gradient is calculated over a batch of training data}

\textcolor{blue}{Since the gradient is calculated on a single example, the error surface will appear noisier than if it was calculated over a batch or the entire training set.}

\textcolor{blue}{When using stochastic gradient descent, it is important to shuffle the data after each epoch.}


% when looking at specific optimizers, http://ruder.io/optimizing-gradient-descent/ was a useful resource

\subsection{Improved Optimizers}

\textcolor{blue}{Some of the common optimizers are listed below. Additional optimizers are discussed in \textcolor{red}{local ref?}}

\subsubsection{Momentum}

\textcolor{blue}{Momentum~\cite{qian1999momentum}, will reduce the learning rate when the gradient is small}

\subsubsection{RMSProp}

\subsubsection{Nesterov}

\textcolor{blue}{Nesterov accelerated gradient (NAG)}

\subsubsection{Adam}

\textcolor{blue}{Adaptive Moment Estimation (Adam)~\cite{kingma2014adam}}


%%%%%%%%%%%%%%%%%%%%%%%%%%%%%%%%%%%%%%%%%%%%%%%%%%%%%%%%%%%%%%%%%%%%%%%%%%%%%%%%%%%%%%
%%%%%% TODO: Cut off -- these optimizers will be moved to another location (research?)
%%%%%%%%%%%%%%%%%%%%%%%%%%%%%%%%%%%%%%%%%%%%%%%%%%%%%%%%%%%%%%%%%%%%%%%%%%%%%%%%%%%%%%

\subsubsection{Nadam}

\textcolor{blue}{Nadam (Nesterov-accelerated Adaptive Moment Estimation)~\cite{dozat2016incorporating}}

\subsubsection{AdaGrad}

\textcolor{blue}{Adagrad~\cite{duchi2011adaptive}, will assign frequently occurring features low learning rates}

\subsubsection{AdaDelta}

\textcolor{blue}{Adadelta~\cite{zeiler2012adadelta}, expands on AdaGrad by avoiding reducing the learning rate to zero.}

\subsubsection{AdaMax}

\subsubsection{Ftrl}

\textcolor{blue}{``follow the regularized leader'', \textcolor{red}{CITE}, \textcolor{red}{works well on wide modes?}}







\chapter{Losses}


\section{losses}

% TODO: not sure where this section belongs
% TODO: It would maybe make sense to have an appendix section that covers the common losses

% potentially ``irrelvant''/~uninterpretible in value to the researcher

\subsection{fit somewhere}

\subsubsection{Contrastive Losses}

\TD{link to secion on self-supervised learning}



\subsection{Overview}

\r{training objective -- continuous}

\subsection{Losses}

\TD{ELBO (Evidence Lower BOund)}

% TODO: index
\TD{Squared logarithmic error (SLE) and Mean SLE (MSLE)}
\TD{Root Mean Squared logarithmic error (RMSLE) and Mean RMSLE (RMSLE)}
\TD{Mean Absolute Percentage Error (MAPE)}


\subsubsection{scalar}

\subsubsection{distribution}


% label smoothing
\subsection{label smoothing}
\TD{Label smoothing}
\TD{When Does Label Smoothing Help? \cite{DBLP:journals/corr/abs-1906-02629}}
\TD{Regularizing Neural Networks by Penalizing Confident Output Distributions \cite{DBLP:journals/corr/PereyraTCKH17}}




\chapter{Genetic Algorithms}

\TD{overview}

%%%%%%%%%%%%%%%%%%%%%%%% Evaluation
\chapter{Evaluation}

\section{Evaluation}

\r{Importance of dataset partitioning \textcolor{red}{local ref?}}

% \textcolor{blue}{The best performance measure will vary depending on the task. For instance, in a medical setting, it may be life threating to classify an event as ``healthy'' when the patient is not healthy.}

\r{A performance measure is used to capture, empirically, how well a prediction made by the model aligns with the expected, ground truth, value.}

\subsection{Creating a Test Set}

% rough para
\r{The most important rule regarding evaluating models, is to ensure that the data used to evaluate the model has never been used before to influence the during training or selection -- this means it was not used during training to update the parameters and it was not used to influence which models are `best' (like a validation set may be used for)}

\r{The performance of a model on a test set may be indicative of how well the model can generalize to unseen data. (This assumes your data sample is representative of the data population)}

\r{Hold-out test set -- created by randomly sampling the dataset. Again, it is important to emphasize that the instances in the test set are never used in the training process and are instead reserved for use only during the evaluation phase.}

\r{peeking\index{peeking}, is an issue that arises when part or all of the test set is included in the training set. This means the model has already seen the data on which the model will be evaluated and so it is possible, probable, that the model will produce high evaluation scores, which will likely translate to an overoptimistic estimation of the models performance when used in production.}

\r{Evaluating the performance of a model can be challenging and will vary depending on the task. For instance, accuracy may not always be the best measure of performance -- consider a medical setting in which sensitivity may be more important since a false negative may be life threatening where as a false positive may only require additional observation.}

\r{When comparing various models, it may be challenging to rank them on a single performance measure. \textcolor{green}{TODO: more.}}

\subsection{Qualitative Evaluation}

\r{generalization is a measure of how well the system preforms on previously unseen data.}

\subsubsection{(Over$|$Under)fitting and Capacity}

\r{{Model capacity}\index{model capacity} helps control how likely a model is to overfit or underfit. Where a model with low capacity may have difficulty fitting a a training set and a model with high capacity may ``overfit'' the data by essentially memorizing the training data.}

\r{Model capcity is closely related to model complexity and the models {hypothesis space}index{hypothesis space} (The set of functions available to the learning algorithm --- \textcolor{green}{TODO: expand - for example a linear vs polynomial model})}

\TD{TODO: figure showing training and validation error and 1) optimal capacity, 2) under and overfitting region 3)generalization gap, 4) capacity}

\paragraph{Overfitting}

\r{Overfitting\index{Overfitting} refers to a case in which a model fits the training data very well but does not fit validation/test set. If a model is overfitting, it is said to have a high variance and is analogous to memorizing the training set.}

\r{Overfitting can arise from modeling data with too many parameters/too complex of a model.}

\TD{TODO: figure showing an example of overfitting}

% addressing overfitting: 1) reduce number of features (manual selection or w/model selection algor) 2) regularization

\begin{figure}[htp]
	\centering
	\includegraphics[width=0.3\textwidth]{example-image-a}\hfil
	\includegraphics[width=0.3\textwidth]{example-image-b}\hfil
	\includegraphics[width=0.3\textwidth]{example-image-c}\hfil
	\caption{Figure example showing the same 2d dataset and an underfitting, overfitting, and ``good'' fitting. \textcolor{green}{TODO} circles=training, x=test -- include scores for each.. slight curve' under=linear, over=extreme poly, good=``smooth''}
	\label{fig:basics_eval_fitting_examples}
\end{figure}

\paragraph{Underfitting}

\r{Underfitting\index{Underfitting} refers to a case in which a model does not fit the training data well. If a model is underfitting, it is said to have a high bias}

\r{Underfitting can arise from modeling data with too few parameters/too simple of a model.}

\TD{TODO: figure showing an example of underfitting}


\subsubsection{Bias Variance Trade-off}

\r{Two fundamental causes of prediction error in a model -- the bias and the variance.}

\paragraph{Variance}
\r{variance\index{Variance} refers to the amount the model would change (consistency or variability) if it was re-trained/estimated multiple using a different subsets of the training data set. A model that has high variance is sensitive to randomness in the training data}

\r{A model with high variance may be described as highly flexible and will likely overfit the data.}


\paragraph{Bias}
\r{Bias\index{Bias} refers to the amount of error that is introduced by approximating a problem with a model that is simpler than the (complex) problem}

\r{A model with high bias will produce similar errors for instances regardless of the training data that is used to train the model -- the model is more strongly ``biased'' to its own assumptions of the relationship (as defined by the model), than the relationship the data may be indicating. A model with high bias may also be described as inflexible and will likely underfit the data.}


% not word-for-word, but example adapted from p35 of ISL
\textcolor{red}{For example, linear regression assumes a linear relationship between the features and labels. However, it is unlikely that a true linear relationship exists and so using linear regression to model this type of particular problem will likely introduce some bias.}

\paragraph{Trade-Off}

% TODO: see page 34 of ISL for eq and explaination here

\r{In general, as a more ``flexible'' model is used, the variance will increase and the bias will decrease.}

\r{One reason to choose a more restrictive model is that they are often more interpretable.}

\begin{figure}[htp]
	\centering
	\includegraphics[width=0.4\textwidth]{example-image-a}\hfil
	\includegraphics[width=0.4\textwidth]{example-image-b}\hfil
	\caption{\TD{side by side figure: a: complex vs simple training on trainnig data (nth poly vs linear), b: same models on test data}}
	\label{fig:basics_eval_tradeoff_examples}
\end{figure}


% see page 36 of ISL
\r{It is easy to obtain a model with low bias but high variance (\emph{e.g.} drawing a squiggly line through every training observation) and it is easy to obtain a model with low variance but high bias (\emph{e.g.} drawing a straight line approximating every training observation) but it is difficult to obtain a model that has both low variance and low bias.}

\textcolor{blue}{It should be noted that in a real world example, it may not be possible to explicitly calculate the test error, bias, or variance.}



\begin{figure}[htp]
	\centering
	\includegraphics[width=0.3\textwidth]{example-image-a}\hfil
	\includegraphics[width=0.3\textwidth]{example-image-b}\hfil
	\includegraphics[width=0.3\textwidth]{example-image-c}\hfil
	\caption{\TD{same 2D dataset with 3 layers and the hidden layer in a has few nodes, b: normal amount of nodes, and c: many nodes} \r{illistrative that the number of connections and complexity increases the chances for overfitting also increases}}
	\label{fig:basics_eval_nodesinhidden}
\end{figure}


\begin{figure}[htp]
	\centering
	\includegraphics[width=0.2\textwidth]{example-image-a}\hfil
	\includegraphics[width=0.2\textwidth]{example-image-b}\hfil
	\includegraphics[width=0.2\textwidth]{example-image-c}\hfil
	\includegraphics[width=0.2\textwidth]{example-image-a}\hfil
	\caption{\TD{same 2D dataset with one, two, three and four hidden layers}}
	\label{fig:basics_eval_numlayers}
\end{figure}

\r{observing a direct trade-off between overfitting and model complexity.}

\r{When we talk about deep learning, we're talking about deep and powerful models that are attempting to solve complex problems that are prone to overfitting and thus usually employ additional countermeasures, such as regularization, to help prevent overfitting.}


\TD{TODO: para about using regularization here/finding the right balance \textcolor{red}{local ref to regularization?}}


%%%%%%%%%%% Metrics (subsec nested under sec.Eval)
\subsection{Metrics}

\textcolor{green}{TODO: talk about metrics/streaming metrics -- mention contrib}

%%%%%%%%%%%%%%%%%%%%%%%% Regularization
\chapter{Improving Generalizability}

\r{The methods shown in the upcoming sections aim to reduce overfitting. That is, these methods aim to prevent the model from becoming too specialized to the training dataset in hopes that it will generalize to data that it has not specifically seen. It is worth noting that by implementing some of these methods (e.g. reducing the model capacity), the model often has less ability to model the training set as well as it might otherwise be able to. This is ok, but worth remembering such that some of the methods aren't used before they are necessary \TD{section on determining overfitting}.}

% TODO: index overfitting
\r{overfitting: a practical definition may include observing the training loss to improve while the validation loss degrades. \TD{possibly mention \\cite{Nakkiran2020DeepDD}}}

\r{Overfitting --- too complex --- Occam's razor --- hypothesis with the fewest assumptions is best}

\r{A specific instance of improving generalization might be accounting for imblance. Either in the labels or in the features.  Section \ref{app_data_imbalance} discusses this topic and strategies in more detail.}

\r{Typicaly types of modifications that are made to improve generalization.}

\begin{itemize}[noitemsep,topsep=0pt]
	\item Data
	\begin{itemize}[noitemsep,topsep=0pt]
		\item Increase ammount of data
		\item Augmentation
		\item Sampling
	\end{itemize}
	\item Architecture --- Reduce complexity of model e.g. applying parameter constraints, and/or reduce overall number of parameters
	\begin{itemize}[noitemsep,topsep=0pt]
		\item Reduce complexity/number of parameters
		\item Ensembling
		\item Constraints
		\begin{itemize}[noitemsep,topsep=0pt]
			\item Directly on parameters
			\item Through additional losses/tasks
		\end{itemize}
	\end{itemize}
	\item Training Pattern
	\begin{itemize}[noitemsep,topsep=0pt]
		\item Early stopping
		\item Stochastic Behavior
	\end{itemize}
\end{itemize}


\section{Data}

\subsection{Data Collection}

\r{Arguably the best way to increase generalizability of a model is to train the model on more data. However, as readers may already be aware, this is not always easy. Collecting more data may not be time/cost effective, or even possible.}

\r{``free'' data in that the ``cost'' is minor computation}

\subsection{Augmentation}

\r{Dataset augmentation is \textcolor{green}{TODO}}

\r{Please note, augmentation must be done responsibly. For example, if performing digit recognition, it would not be wise to perform rotational or flip transformations on the data since, depending on the specific data, a 6, rotated 180 or flipped vertically may now appear as a 9.}


\r{invariances in the data}

\r{For specific techniques, see~\ref{app_aug_techniques}}

\TD{Beyond improving generalization, augmentation may be used in other contexts as well, such as in helping quantify uncertainty -- \TD{see ref ---\TD{Augmenting the test set. A simple augmentation (horizontal filliping) was performed on the test set in \cite{simonyan2014very} -- where the prediction of the original and augmented images are averaged to obtain the final output score.} }}


\subsection{Sampling}

\r{The line between the techniques described here and ``augmentation'' might be a little blurred, in that sampling might technically be considered a augmentation technique (and I'm not even sure ``sampling'' is the appropriate title). But the intended distinction is that in augmentation, we are diliberately altering something (e.g. the input data) and in sampling, we are altering the number of times an architecture sees a particular instance in a training dataset.}



\section{Architecture}

\section{Training Pattern}

\subsection{Early Stopping}

\r{see p.243 of DL, papers Bishop 1995 and Sjoberg and Ljung 1995}

% TODO: note about regularization --- the smaller the value, the stronger the regularization.


\subsection{Stochastic Behavior}

\subsubsection{Dropout}

\r{``Dropout'' as a node in a computational graph may be considered an architectural structure change, but the method itself affects the training pattern in possibly not obvious ways. }

% TODO: explain dropout

\r{Dropout -- ref original paper (Hinton? -- intuitive, inspired by bank -- that defrauding the bank would require cooperation between employees to defraud the bank \TD{cite})}.

\r{Dropout (proposed in ``Improving Neural Networks by Preventing Co-Adaption of Feature Dectors''~\cite{DBLP:journals/corr/abs-1207-0580}, and popularized by Nitish et.al in ``Dropout: a Simple Way to Prevent Nerual Networks from Overfitting''~\cite{JMLR:v15:srivastava14a}}

\r{It is important to note that dropout is only present during training. i.e. dropout does not occur during test/evaluation if using dropout in the ``standard way''. However dropout is occassionally used for evaluation in attempt to quantify model uncertainty \TD{CITATION}}

\r{keeps a neuron active by a hyperparameterized probability.}

\r{used in any/all neurons in the network (other than the output neruons).}

\r{think about where dropout is used. That is when you use dropout at any given nueron the upstream paths transversing that particular neuron are also affected (in this case, ``turned off''), as well downstream connections (but often only modified, not entirely turned off since they often still have other inputs) }

\r{Forces the network to learn mappings even in the absence of all the information, that is the network is forced to consider the values of other values and can't rely on a smaller number of values or groups of values. Said another way, the network is prevented from becoming too dependent on certain inputs or features.}

\r{In this way, dropout can be thought of as sort of an ensembling method. When dropout is in use during training, each loop technically produces a different network that is then trained for the given task. During the next loop, a different network is used. As Aurélien Géron~\cite{geron2019hands} describes, if you train for 10,000 training steps (where dropout is used), you will have likely (almost certainly) trained 10,000 different neural networks. It's true that each network is not indpendant (they share weights), but they are different. More generally, a network with $N$ activations with dropout present, there exist $2^N$ possible networks ($2$ since each activation/neuron/value can have either an `on` or `off` state.) and thus, the use of all of these networks at once can be considered an ensembling of sorts.}

\TD{create figure of this ensemble of many networks.}


% TODO: find recent paper I saw mentioned on twitter.... (4July) it may be in my pocket

\begin{figure}[htp]
	\centering
	\includegraphics[width=0.3\textwidth]{example-image-a}\hfil
	\includegraphics[width=0.3\textwidth]{example-image-b}\hfil
	\includegraphics[width=0.3\textwidth]{example-image-c}\hfil
	\caption{\TD{Graph of an example function including dropout. three separate training iterations and how the network changes}}
	\label{fig:regularization_dropout_overview_training}
\end{figure}

\begin{figure}[htp]
	\centering
	\includegraphics[width=0.3\textwidth]{example-image-a}\hfil
	\caption{\TD{Same graph during test --- no dropout applied}}
	\label{fig:regularization_dropout_overview_test}
\end{figure}

\r{It is worth pointing out that since dropout is only applied at training time, comparing the loss curve of training and inference (validation splits) will be a bit misleading since the full ensemble network is used for calculating the validation loss/metrics and only the component \TD{is there a better word than this?} networks are used for the training set.}

\r{Additionally, if you run the training set through multiple times, you may find slightly different results. Again, this is because while dropout is on, you'll find that a slightly different network is used. \TD{This idea can be exploited at inference time to get uncertainty estimates.}}

\r{some important notes about the implementation. The outputs at test time should be equivalent to their expected outputs at training time (which is altered due to the application of dropout).}

\r{Couple solutions}
\begin{itemize}[noitemsep,topsep=0pt]
	\item scale the outputs during inference
	\item
\end{itemize}

\r{One potential solution to this problem is to scale the outputs during inference in a way that compensates for the dropout probability.  For example, if the dropout rate was set to $0.5$, then it would become necessary to halve the neurons outputs at test time in order to keep the expected output the neurons have learned during training.  However, this may not be ideal in practice since it would require scaling all the neuron outputs at test time (where performance is often critical and more important).}

\r{at test time, multiply the values by the expectation, not the on/off mask}

\r{Another, perhaps more desirable solution, would be to use \IDI{inverted dropout}. The cs231n~\cite{cs231n} course provides a concise explaination and example code on this topic.}

\r{This applies the same principal as outlined above, only the scaling occurs at training time rather that at test time. That is, during training, any neuron whose activation was not turned off, has the output divided by the dropout rate before being propagation to the next layer.  This way, at test time, no scaling is required.}


\TD{DropConnect~\cite{wan2013regularization} is similar to dropout, except that individual weights are disabled, not entire individual nodes and can be considered a generalization of dropout.}

\TD{figure showing difference}

% `drop block''?
\TD{investigate more structured dropout.}

% helps learn ``multiple paths''/simulates ensembles

\TD{alpha dropout\cite{DBLP:journals/corr/KlambauerUMH17}}

\TD{link to ensemble section}

\subsection{Parameter Regularization}

\r{Collection of techniques used to help generalize a model -- which may help prevent overfitting. Typically regularization penalizes complexity of a model.}


% TODO: figure of loss plot showing a steep training and shallow+divergent val/test loss

\r{Helps prevent the model from memorizing noise in the training data.}


\subsubsection{Types of Regularization}

\textcolor{blue}{Regularization is an active area of research.}

% more information on L1/L2 http://www.chioka.in/differences-between-l1-and-l2-as-loss-function-and-regularization/

\begin{itemize}[noitemsep,topsep=0pt]
	\item Early Stopping (implementation: \textcolor{red}{local ref})
	\item Parameter Norm Penalties (implementation: \textcolor{red}{local ref})
	\begin{itemize}[noitemsep,topsep=0pt]
		\item L1 (Lasso) Regularization
		\item L2 (Ridge) Regularization
		\item Elastic Nets
	\end{itemize}
	\item Dataset Augmentation (implementation: \textcolor{red}{local ref})
	\item Noise Robustness
	\item Sparse Representations
	\item Dropout (implementation: \textcolor{red}{local ref})
	\item Ensemble methods (implementation: \textcolor{red}{local ref})
	\item Adversarial Training
\end{itemize}



\subsubsection{Parameter Norm Penalties}

\r{key difference is the penalty term}

\TD{TODO: DIGRAM OF L2 + L1 + elastic nets}

\paragraph{L2 Regularization}

\TD{TODO: DIAGRAM OF L2}

\r{L2, ({Ridge regression}\index{Ridge regression}) may also be known as {Tikhonov regularization}\index{Tikhonov regularization}}

\r{penalizes model parameters that become too large. Will force most of the parameters to be small, but still non-zero}

\r{square of the absolute value of the coefficient}

\begin{figure}[htp]
	\centering
	\includegraphics[width=0.3\textwidth]{example-image-a}\hfil
	\includegraphics[width=0.3\textwidth]{example-image-b}\hfil
	\includegraphics[width=0.3\textwidth]{example-image-c}\hfil\\
	\medskip
	\includegraphics[width=0.3\textwidth]{example-image-a}\hfil
	\includegraphics[width=0.3\textwidth]{example-image-b}\hfil
	\includegraphics[width=0.3\textwidth]{example-image-c}\hfil
	\caption{\TD{Top: NN output decision boundary on 2D dataset Bottom: weight params distribution from tensorboard... from LtoR = same arch with varying degrees of L2 regularization (0.01, 0.1 and 1.0)}}
	\label{fig:basics_regularization_l2_example}
\end{figure}


% p91(71) of mastering ML w SKL says "when lambda is equal to zero, ridge regression is equal to linear regression"

\paragraph{L1 Regularization}

\TD{TODO: DIAGRAM OF L1}

\r{LASSO (\textbf{L}east \textbf{A}bsolute \textbf{S}hrinkage and \textbf{S}election \textbf{O}perator) --- produces sparse parameters. This will force coefficients to zero and cause the model to depend on a small subset of the features.}

\r{absolute value of the weight coefficient}

\r{use only a small subset of the input features and can become resistant to noisy inputs.}

\r{It could be argued that using L1 regularization may help to make a model more interpretable, by using less (presumably more important/relevant) features when making predictions.}

\r{The use of L1 regularization for feature selection}


\paragraph{Elastic Net Regularization}

\r{Linearly combines the $L^1$ (feature selection) and $L^2$ (generalizability) penalties used by both LASSO and ridge regression. The cost is having two parameters (as opposed to just one when using either L1 or L2).}

\TD{TODO: figure}.



\subsection{Ensemble Methods}

\r{see \textcolor{red}{local ref} for more information on ensemble basics and see \textcolor{red}{local ref} for implementation details.}

% TODO: find Breiman 1994 paper referenced in p249 of Deep Learning
\r{As described in \textcolor{red}{local ref} ensemble methods act as a form of regularization by combining several different models \TD{Breiman 1994}. This often improves generalizability since the included models will often make independent, different, errors on the data.}

\subsection{Adversarial Training}



\subsection{Transfer Learning}

% TODO: is this talked about anywhere else? this is probably the best place for it.

\TD{TODO: transfer learning, using -- explanation}

\TD{tool that may sometimes be efficient way of getting to potentially more accurate approximations, faster. \TD{citations}}

% TODO: index
\r{using parameters or pre-trained components from a model/task for a new model/task.  Process of adapting these components to a new model/task is called fine-tuning}


\begin{figure}[htp]
	\centering
	\includegraphics[width=0.5\textwidth]{example-image-a}\hfil
	\caption{Figure example layer hierarchy and where/when to transfer/freeze params -- this will be 1-2 figures and include many sub-figures \textcolor{green}{TODO}}
	\label{fig:transfer_learning_subfigs_a}
\end{figure}

\textcolor{green}{{freezing}\index{freezing} parameters or a layer means preventing the parameters from being updated during training. This is often controlled by a parameter called ``trainable'' see \textcolor{red}{local TF ref} for the TensorFlow implementation and \textcolor{red}{local YF ref} for the YamlFlow implementation. In relation to transfer learning and freezing, mention the difficulty of propagating updates though a large network}

% TODO: this likely does not belong here...
\subsubsection{Normalization}

\TD{TODO: overview para + importance}

\TD{TODO: figure showing differences}

\paragraph{Instance normalization}

\r{see section in preprocessing \textcolor{red}{local ref?}}

\paragraph{Layer normalization}

\paragraph{Batch normalization}

\TD{Show / explain}

\TD{Batch Normalization: Accelerating Deep Network Training by Reducing	Internal Covariate Shift \cite{DBLP:journals/corr/IoffeS15}}

\r{similar to dropout \ALR, the behavior of batch norm is different at training time and inference time.}

\r{Standard implementation is to calculate the population values using an exponential moving average (EMA).}

%TODO: here!

\TD{{Rethinking "Batch" in BatchNorm}~\cite{Wu2021RethinkingI} concludes that using EMA as the method for calculating the population statistics is not ideal. They show that during the early epochs, the xxxxxxx.}

\r{An adaptive re-parameterization.}

\r{reduce sensitivity to hyperparameterization.}

\TD{TODO: transfer learning considerations --- will likely have to unfreeze these params}

% HUGO talk
\r{``making the optimization easier''. batch norm is not effective in RNNs -- more so layer norm}

\r{seems to help when both under and over fitting.}

\r{order: pre-act, then batch norm, then activate.}

\r{$\gamma$ and $\beta$ parameters that are learned parameters. These params could effectively undo the normalization caused (if ``learned'' to do so.)}

\begin{enumerate}[noitemsep,topsep=0pt]
	\item batch statistics
	\begin{itemize}[noitemsep,topsep=0pt]
		\item mean
		\item variance
	\end{itemize}
	\item normalize the pre-activation
	\item $\gamma$ and $\beta$ --- learned rescalling
\end{enumerate}


% Graham Taylor talk
\begin{itemize}[noitemsep,topsep=0pt]
	\item turn down other regularization
	\item fixes first and second moments which may suppress information in these moments.
\end{itemize}

\TD{work related to adversarial spheres. --- with batch norm, the result was more reflective of the batch, not the entire dataset (which makes sense, right?)}


\paragraph{Group normalization}


\section{Output regularization}

\r{confidence penalty on predictions that are extrememly confident\cite{pereyra2017regularizing}. Originally an RL idea to promote expoloration. In SL, we would prefer fast convergence i) anneal confidence penalty ii) only penalize at a certain confidence threshold (lower entropy threshold). Intuitive (or not), can improve generalization.}

%TODO:
\r{label smoothing\cite{szegedy2016rethinking}}

\r{Adding label noise\cite{xie2016disturblabel}}

\r{smooth labels -- either via a ``teacher model''\cite{hinton2015distilling} or using it's own distribution\cite{reed2014training}}


\r{virtual adversarial training\cite{miyato2018virtual}}


%%%%%%%%%%%%%%%%%%%%%%%% Distributed
\textcolor{blue}{see \textcolor{red}{local ref} for more detailed information about distributed training/evaluation.}


\subsection{Resource Management (GPU RAM)}
% TODO: implementation

\textcolor{blue}{By default, TensorFlow will allocate, and hold, all available GPU resources. This means if using \code{/GPU:0}, that 100\% of the resources will be dedicated to this program and so another tensorflow program cannot access this resource. Additionally, this means that if there are multiple GPU devices 100\% of those devices will also be allocated. If attempting to run two programs on the same device, with default settings, the following error will likely be encountered \code{CUDA\_ERROR\_OUT\_OF\_MEMORY}. This behavior can be controlled by either setting the device visibility or setting a device allocation threshold}

\paragraph{Device Visibility}

\textcolor{blue}{The environment variable \code{CUDA\_VISIBLE\_DEVICES} can be set to define specific device access for a program. e.g. \code{CUDA\_VISIBLE\_DEVICES=0,1} would allow the program access to only GPU devices 1 and 2.  Another program could then be started with \code{CUDA\_VISIBLE\_DEVICES=2,3} that would restrict it's access to GPUs 3 and 4}
\textcolor{blue}{For Example, if we only wanted the 3rd GPU to be accessed by a specific program, we could do the following (remember 0 based indexing):}

\begin{lstlisting}[style=pyInStyle]
import os
...
os.environ['CUDA_VISIBLE_DEVICES'] = '2'
\end{lstlisting}



\paragraph{Device Allocation}

\textcolor{blue}{It is also possible to only allow access to a specified fraction of a device i.e. n\% of a GPU.  This can be managed by altering the \code{ConfigProto} object. The \code{ConfigProto} object contains, among other configuration defaults (as messages), a message called \code{GPUOptions}. When running a session, these defaults can be changed. Within \code{GPUOptions}, there is a variable \code{per\_process\_gpu\_memory\_fraction} that is responsible for allocating a fraction of GPU memory.}

\textcolor{blue}{For example, if we wish to only allocate 40\% of a GPUs memory, we could do the following:}

\begin{lstlisting}[style=pyInStyle]
myConfig = tf.ConfigProto()
myConfig.gpu_options.per_process_gpu_memory_fraction = 0.4
session = tf.Session(config=myConfig)
\end{lstlisting}


\subparagraph{Allow Growth}

\textcolor{blue}{There is another option when managing the device memory. Rather than set a hard threshold, it is possible to allow TensorFlow to only allocate as much memory as it needs. This can be done by setting the \code{allow\_growth} bool to \code{True}. However, use caution with this option! \textbf{Once TensorFlow allocates memory, it is never released \textcolor{red}{``to avoid memory fragmentation''} i.e. running out of memory later is possible}}

\begin{lstlisting}[style=pyInStyle]
myConfig = tf.ConfigProto()
myConfig.gpu_options.allow_growth = True
session = tf.Session(config=myConfig)
\end{lstlisting}


\subsection{Device Placement}

\textcolor{blue}{A \textit{dynamic placer} algorithm is presented in the {TensorFlow whitepaper}~\cite{abadi2016tensorflow_device_placement} that is capable of automatically distributing operations across devices. This algorithm takes into consideration  estimates of the sizes of input and output tensors, computation time for each node, \textcolor{green}{OTHERS - TODO:read entire paper}.}

\textcolor{blue}{There is another (manual) placer called a \textit{simple placer}.}

% TODO: MORE...

% TODO: implementation



