\chapter{Prerequisites}

\textcolor{blue}{The following sections are a non-exhaustive, brief, refresher on some of the important underlying concepts and methodologies used by later concepts. Resources to further explore and learn these concepts are shared.}


\section{Math Notation}

\textcolor{blue}{Depending on the resource, the level of formal math education required to understand a passage may vary greatly. In order to demystify some of the resources that do not expand on the proofs and notations, below are some of the symbols and XXXXX used in math notation and their interpretation in simple text/natural language form.}

\textcolor{green}{TODO: Show symbols and examples in both equation and simple text/natural language form.}

\textcolor{blue}{$\mathbb{R}$}


\section{Boolean Logic}

\textcolor{green}{TODO: background/overview on boolean logic and importance}

\textcolor{green}{TODO: Examples}


\section{Linear Algebra}

\subsection{Overview}

\textcolor{green}{TODO: background/overview on linear algebra and importance}

\textcolor{green}{TODO: Examples}

\textcolor{blue}{Matrix -- written as (rows x columns)}

\textcolor{blue}{$M_{i,j}$ means matrix entry at ($i$,$j$), or $i$th row, $j$th column}

\textcolor{blue}{vector: an $n x 1$ matrix. A $12 x 1$ vector may be considered a 12 dimension vector.}

\textcolor{blue}{1-indexed or 0-indexed. In this document, we will only use 0-indexed terms}

\textcolor{blue}{Matrices -- denoted by uppercase names}

\textcolor{blue}{vectors -- denoted by lowercase names}

\textcolor{blue}{scalar -- single value (not a vector or matrix)}

\subsection{Matrix Arithmetic}

\subsubsection{Matrices}

\paragraph{Addition, Subtraction}

\paragraph{Multiplication, Division}

\subsubsection{Scalar}

\paragraph{Addition, Subtraction}

\paragraph{Multiplication, Division}



\section{Graph Theory}

\textcolor{green}{TODO: Examples}