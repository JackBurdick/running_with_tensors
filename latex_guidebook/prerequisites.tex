\chapter{Prerequisites}

\textcolor{blue}{The following sections are a non-exhaustive, brief, refresher on some of the important underlying concepts and methodologies used by later concepts. Resources to further explore and learn these concepts are shared.}

\textcolor{blue}{But first, let's start by answering the million dollar question: what the heck is a tensor?  A {tensors}\index{tensor} are a generalization of matrices and can have an arbitrary number of dimensions (axis). The number of axes tensors of a tensor is also called the {rank}\index{rank}.}


\section{Math Notation}

\textcolor{blue}{Depending on the resource, the level of formal math education required to understand a passage may vary greatly. In order to demystify some of the resources that do not expand on the proofs and notations, below are some of the symbols and XXXXX used in math notation and their interpretation in simple text/natural language form.}

\textcolor{green}{TODO: Show symbols and examples in both equation and simple text/natural language form.}

\textcolor{blue}{$\mathbb{R}$}

\section{Calculus}

\textcolor{yellow}{Don't worry we won't be proving any theorems and I won't be assigning 45 questions due tomorrow. But, there is some basic terminology that should be revisited.}

\textcolor{blue}{{Critical points}\index{Critical points} or {stationary points}\index{stationary points} are points where the derivative is equal to zero and therefore doesn't provide any useful information about the gradient/slope (which direction and how far to move)}

\begin{figure}
	\centering
	\includegraphics[width=0.5\textwidth]{example-image-a}\hfil
	\caption{Graph of an example function and its derivative, \textcolor{green}{TODO}}
	\label{fig:calc_fn_deriv}
\end{figure}
\textcolor{green}{TODO: graph of function and it's derivative overlaid.}

\textcolor{blue}{There exist three main types of critical points:}

\begin{itemize}
	\item \textcolor{blue}{local minimum} -- 
	\item \textcolor{blue}{local maximum} -- 
	\item \textcolor{blue}{saddle point} -- 
\end{itemize}

\begin{figure}[htp]
	\centering
	\includegraphics[width=0.3\textwidth]{example-image-a}\hfil
	\includegraphics[width=0.3\textwidth]{example-image-b}\hfil
	\includegraphics[width=0.3\textwidth]{example-image-c}
\caption{Types of critical points -- points with zero slope. From left to right, \textcolor{green}{TODO}}
\label{fig:calc_critical_points}
\end{figure}


\textcolor{blue}{Functions may have many local minimal and plateaus. The point at which the absoluet lowest value is considered the {global minimum}\index{global minimum}. Similarly a value located at the largest absolute value of the function is considered the {global maximum}\index{global maximum}. }

\begin{figure}
	\centering
	\includegraphics[width=0.5\textwidth]{example-image-b}\hfil
	\caption{Graph of an example function with many (labeled) local minima and plateaus. the global minima should also be labeled. \textcolor{green}{TODO}}
	\label{fig:calc_fn_deriv}
\end{figure}

\subsection{Chain Rule}

\textcolor{blue}{(not the chain rule of probability)}



\section{Boolean Logic}

\textcolor{green}{TODO: background/overview on boolean logic and importance}

\textcolor{green}{TODO: Examples}


\section{Linear Algebra}

\subsection{Overview}

\textcolor{green}{TODO: background/overview on linear algebra and importance}

\textcolor{green}{TODO: Examples}

\subsubsection{Scalars (0D tensors)}

\textcolor{green}{TODO: diagram}

\textcolor{blue}{scalar -- single value (not a vector or matrix)}

\textcolor{green}{example: a single value}



\subsubsection{Vectors (1D tensors)}

\textcolor{green}{TODO: diagram}

\textcolor{blue}{vectors -- denoted by lowercase names}

\textcolor{blue}{vector: an $n x 1$ matrix. A $12 x 1$ vector may be considered a 12 dimension vector.}

\textcolor{blue}{1-indexed or 0-indexed. In this document, we will only use 0-indexed terms}

\textcolor{green}{example: a ``list'' of features}


\subsubsection{Matrices (2D tensors)}

\textcolor{green}{TODO: diagram}

\textcolor{blue}{Matrix -- written as (rows x columns)}

\textcolor{blue}{$M_{i,j}$ means matrix entry at ($i$,$j$), or $i$th row, $j$th column}


\textcolor{blue}{Matrices -- denoted by uppercase names}

\textcolor{green}{example: a grayscale (single color channel image)}



\subsection{Matrix Arithmetic}

\subsubsection{Matrices}

\paragraph{Addition, Subtraction}

\paragraph{Multiplication, Division}

\subsubsection{Scalar}

\paragraph{Addition, Subtraction}

\paragraph{Multiplication, Division}



\section{Graph Theory}

\textcolor{green}{TODO: Examples}

\textcolor{blue}{computational graph}

\textcolor{blue}{operation}

