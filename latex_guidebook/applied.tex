
\chapter{Experimental Design}

\section{Overview}

\TD{\ALR stats section where needed}


\TD{Assumption here is that metrics and business usecase have been defined --- which should not be underestimated}
\TD{Design}
\begin{itemize}[noitemsep,topsep=0pt]
	\item Power analysis --- how big does the sample need to be \ALR
	\item Sample selection (\textit{e.g.} is your sample a user, time of day, group of users, type of order, filters needed? etc) \ALR
	\item $<$Experiment$>$
	\item Experimental Analysis \ALR
\end{itemize}

\TD{A/B tests and analysis on experiments is a noisy process \TD{link to book}}

\r{To increase the statistical power, the most straightforward methods may be to increase the amount of data. This may include running the experiment for longer or including more samples. However, due to a number of constraints, this may not always be possible.}

\r{However, there are other approaches that may be used}

\begin{itemize}[noitemsep,topsep=0pt]
	\item Stratification \ALR
	\item Control variates \ALR
\end{itemize}


\chapter{Model Reporting}

\TD{Model Cards for Model Reporting \cite{DBLP:journals/corr/abs-1810-03993}}

\chapter{Deployment}


\chapter{Monitoring}

% TODO: TFX

