\chapter{Term dump}

\textcolor{green}{Terms that are important but haven't been placed in the document yet}

\emph{Collinearity} --- When two or more predictor variables are closely related to one another they are said to be collinear.

\emph{Curse of Dimensionality} --- \r{phenomenon where the feature space becomes increasingly sparse as the number of dimensions/features is increased -- trade off between the density of instances in the feature space and the number of dimensions (number of descriptive features) \r{including too many features can paradoxically, lead to worsening of performance.} \textcolor{green}{TODO: mentioned in paper Bellman 1961}}

\emph{dummy variable} ---


\emph{Population vs Sample} -- the population (usually denoted $N$) is the collection of all the items of interest in a study where as the sample is a subset of a population (usually denoted $n$). The numbers obtained when working with a population are called the `parameters' and the numbers obtained when working with a sample are a called `statistics'. \textcolor{blue}{a random sample is obtained when each member of the sample is chosen from the population by chance and accurately reflects the population}

\emph{$e$} --- \textcolor{blue}{Euler's numbers}

\subsection{Distributions}

\emph{Normal Distribution} --- \textcolor{blue}{Normal, or Guassian, distribution. Data is symmetrical where half the values are greater than the mean and half the values are less than the mean. The median, mode, and mean are all equal}

\emph{Bernoulli Distribution} --- \textcolor{blue}{distribution in which XXXXXXXXXXXX}

\emph{epoch} --- \textcolor{blue}{a complete pass through the entire training set.}

\emph{vector} --- \textcolor{blue}{direction, and magnitude (length)}

\emph{eigenvalue} and \emph{eigenvector} --- \textcolor{blue}{`eigen' is a german word for ``belonging to'' or ``particular to''.} 