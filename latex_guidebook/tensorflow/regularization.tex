\subsection{Regularization}

\textcolor{blue}{regularization is discussed in \textcolor{red}{local ref?}}

\subsubsection{Early Stopping}

% TODO: implementation

\subsubsection{Parameter Norm Penalties}

\textcolor{blue}{weights + biases}

\paragraph{L2 Regularization}

% TODO: implementation

\paragraph{L1 Regularization}

% TODO: implementation

\paragraph{Elastic Net Regularization}

% TODO: implementation


\subsubsection{Dataset Augmentation: Image Augmentation}

\textcolor{blue}{Image augmentation can be performed in TensorFlow using the image API.}

\paragraph{Resizing and Cropping (Zooms, Translation)} 

% TODO: implementation


\paragraph{Flipping, Rotating, and Transposing}

% TODO: implementation

\paragraph{Colorspaces}

% TODO: implementation

\paragraph{Image Adjustments}

% TODO: implementation

\paragraph{NOTE: Other Libraries}

\textcolor{blue}{The focus of this section is on using TensorFlow to create image augmentations. However, there are other types of augmentations that may be useful. These augmentations, though possible to be produced in tensorflow, are not \textit{easily} produced and so other libraries may be useful to perform any of the following augmentations. \textcolor{red}{however, please note that the conversion between tensor to value to tensor may be necessary and may be time consuming. Additionally, it will be necessary to ensure that the implementation considers whether it is performing the manipulations on the data or on a copy of the data. This can be really important when dealing with destructive methods (such as zooms where some of the image information is removed) since, if successive zooms are performed on the same image (not a copy), after several iterations the image may be not representative of the true class \textcolor{green}{TODO: example}}}

\textcolor{green}{TODO:}

\textcolor{blue}{It should be noted that keras has built-in functions that perform many of these augmentations when producing a dataset. I recommend viewing the documentation regarding these methods.}

\subsubsection{Dropout}

% TODO: implementation


% TODO: this may not belong here...
\subsubsection{Normalization}

\textcolor{blue}{TODO: overview para + importance see \textcolor{red}{local ref to basics}}

\paragraph{Instance normalization}

\textcolor{blue}{see section in preprocessing \textcolor{red}{local ref?}}

% TODO: implementation

\paragraph{Layer normalization}

% TODO: implementation

\paragraph{Batch normalization}

% TODO: implementation

\paragraph{Group normalization}

% TODO: implementation