\subsection{Data}

\subsubsection{Reading in Data}

\paragraph{Different file formats}
\textcolor{blue}{Datasets can be created from different file formats}

% tf.data.Dataset.TextLineDataset
% tf.data.Dataset.TFRecordDataset
% tf.data.Dataset.FixedLengthRecordDataset
% otherwise, it is also possible to add personal decoding code

% TODO: 
\subparagraph{tf.data.Dataset.TextLineDataset}

% TODO: example loading in .csv file

\subparagraph{tf.data.Dataset.TFRecordDataset}

% TODO: example loading in tf record file

\subparagraph{tf.data.Dataset.FixedLengthRecordDataset}

% TODO: example loading in FixedLengthRecordDataset file

% NOTE: input\_fn is called only once, at model creation time, not every time it needs data. Also, they do not return data, they return tensorflow nodes. SO, make sure the input function is expected to only run once and that the operations are expressed in tensorflow operations


\paragraph{Reading from Sharded Files}

% 1. list\_files("glob/like/pttern*")
% 2. flat\_map(tf.data.TextLineDataset)
% 3. map(custom_decode_line_fn)

% flatmap vs map.. one to one transformations (parsing line of text into an int) (map) and one to many transformations (one filename becomes a collection of text lines) (flat\_map)


\subsubsection{Transforming Data}