\chapter{Foundational Methods}

\section{Regression}

\subsection{Simple Linear Regression}


\begin{equation}
{Y \approx \beta_0 + \beta_1 X}
\label{eq:slr_ex}
\end{equation}

\textcolor{blue}{$\approx$ can be read as ``\emph{is approximately modeled as}''. $Y$ is a quantitative response (output/prediction) and $X$ predictor variable(input/feature). $\beta_0$ and $\beta_1$ are two unknown constants representing the intercept and slope, respectively. These unknown values that determine the behavior of the model are known as the model \emph{parameters} or \emph{coefficients}}

% see p62 of ISL for more

\subsection{Multiple Linear Regression}

\textcolor{blue}{Using $n$ predictors:}

\begin{equation}
{Y \approx \beta_0 + \beta_1 X_1 + \beta_2 X_2 + \cdots + \beta_n X_n}
\label{eq:mlr_ex}
\end{equation}


\subsection{Polynomial Regression}



\subsection{K-Nearest Neighbors}

\textcolor{blue}{The optimal value for k will depend on the bias-variance trade-off}

\section{Classification}

\subsection{Logistic Regression}

